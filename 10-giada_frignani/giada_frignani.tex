% !TEX encoding = UTF-8
% !TEX program = xelatex
\documentclass[12pt,a4paper]{article}
\usepackage[paperwidth=210mm, paperheight=297mm, left=0.75in, right=0.75in, bottom=1in, top=1in]{geometry}
\usepackage{polyglossia}
\setdefaultlanguage[babelshorthands]{italian}
\usepackage{fontspec}
\usepackage{graphicx}
\usepackage{blindtext}
\usepackage{wrapfig}

\frenchspacing
\makeindex

\begin{document}
\title{\vspace{-70pt}Telescopio spaziale AMS}
\author{Giada Frignani}
\date{}
\maketitle
\pagestyle{empty}
\thispagestyle{empty}

\section*{Storia}
\label{storia}
\begin{wrapfigure}{r}{0.35\textwidth}
  \vspace{-10pt}
  \begin{center}
    \includegraphics[width=0.30\textwidth]{satellite}
  \end{center}
  \vspace{-20pt}
\end{wrapfigure}
Mercoledì 3 Aprile 2013, è di portata storica per la Scienza. È il primo successo scientifico di AMS, il più potente e sensibile spettrometro magnetico galattico iper-freddo, appositamente concepito, costruito e messo in orbita sulla Stazione Spaziale Internazionale (Iss) a 400 Km di quota, per la ricerca di Antimateria e Materia Oscura nel Cosmo.

\section*{Osservazioni}
\label{osservazioni}

\emph{AMS--02 è uno spettrometro magnetico di grandi dimensioni realizzato per operare nello spazio. Tra le tante sfide del progetto, la principale è stata quella di mettere a punto un sistema magnetico in grado di lavorare nello spazio in sicurezza e per un periodo prolungato}.

La Collaborazione di AMS ha sviluppato due magneti:

\begin{itemize}
\item Un \textbf{magnete permanente} (Permanent Magnet, PM) che opera a temperatura ambiente.

\item Un \textbf{magnete superconduttore} (Superconducting Magnet, SCM) in grado di operare a una temperatura di 4 gradi sopra lo zero assoluto (0 K).

\end{itemize}

\textbf{L’elettronica} di AMS è una sfida tecnologica a sè stante. Lo spazio è infatti un ambiente estremamente ostile per le apparecchiature elettroniche data la presenza della radiazione cosmica. L’elettronica di AMS consta di oltre 600 diversi computer dotati di chip tolleranti alla radiazione, sviluppati appositamente per la fisica delle alte energie, circa 10 volte più veloci dei normali computer.

I risultati acquisiti da AMS, perfettamente in accordo con i modelli matematici e fisici più accreditati nel Modello Standard, sono fondati sull’osservazione di 25 miliardi di eventi che includono il rilevamento di 400mila positroni, di energia compresa tra 0.5 e 350 GigaelettronVolt (GeV), sul totale di 6.8 milioni di particelle osservate nell’intervallo energetico, nel corso del primo anno e mezzo di attività, dal 19 Maggio 2011 al 10 Dicembre 2012. È la prima scoperta da World Guinness Record dell’esperimento AMS. I dati acquisiti, infatti, rappresentano la più estesa collezione di particelle di Antimateria mai osservata e registrata prima nello spazio. Il primo “set” di positroni incrementa considerevolmente nell’intervallo di energia tra 10 e 250 GeV, con una curva piuttosto pronunciata di dati che poi diminuisce di un ordine di grandezza al di sopra del range 20--250 GeV. I dati non mostrano sostanziali variazioni significative nello spaziotempo: il flusso energetico di anti-particelle giunge da ovunque lo si osservi senza direzioni preferenziali. I risultati di AMS sono consistenti e in buon accordo, secondo gli scienziati, con il comportamento dei positroni prodotti dall’annichilazione di particelle di Materia Oscura nello spazio cosmico. Ma i dati finora acquisiti non sono ancora sufficienti per escludere altre spiegazioni.

\end{document}